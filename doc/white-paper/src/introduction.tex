In the age of hypercapitalism, wealth inequality has become a growing and 
prominent issue. The rise of the internet provided people the means of 
constructing a new reality in which they can be free of social division and the 
constructs fabricated by the establishment while freely expressing themselves. 
In this new reality each individual is equal and has access to the same 
limitless resources. This did not last long however, as more businesses 
discovered the opportunity of extracting wealth from this new reality. This 
demonstrated advantage of significantly increasing the amount of content and 
services available. However, the provided software and media is hidden behind 
pay walls and region locking mechanisms. The application of the free market - 
with all its disadvantages (like being prone to abuse) - in the world of the 
internet is even less justified because we cannot speak about the concept of 
scarcity in this context, one being able to replicate information indefinitely. 
Sharing copyrighted content became the norm for a while, in this way bridging 
the gap between the low and high income earners \cite{piracyinequality}, but 
more aggressive policies crept in the world of the internet trying to regulate 
and punish the free flow of digital information.

Internet \textit{pirates}\footnote{Person who downloads and distributes 
copyrighted content digitally without permission, such as music or software 
\cite{CHOI2007168}} discovered ways to circumvent sanctions from the authority 
by using different means, each with its own disadvantages. Private torrent 
trackers are unaccessible to any legitimate pirate, while VPN services are not 
free and one should be skeptical about the providers of these services. The Tor 
anonymity protocol is also great, but has the downside of having high latency 
and being prone to network congestion \cite{torcongestion}. Furthermore, piracy 
is considered unethical over the Tor network because it might slow down more 
important communication between individuals living in autocratic regimes with 
high degree of censorship, which rely on anonymous communication on a daily 
basis.

The SAND protocol (recursive acronym for \textit{SAND Anonymous Distribution}) 
leverages the power of a worldwide P2P network, in which each node can act as 
proxy for another node to provide anonymity in communication. The primary 
advantage over the BitTorrent protocol is that the identities of the two 
parties (uploader and downloader) are unknown to each other. This is achieved 
by the means of a file searching algorithm that recursively propagates a 
request through the network while probing each node until the file is found. 
The chosen solution tries to attain the optimal balance between network 
overhead and security. Each node is a volunteer in the network, as a part of 
the bandwidth will have to be donated for the benefit of the collective while 
the client application is running.
