The proposed system is a practical way of distributing large file anonimously 
among peers on the internet. It is easy for everyone to join the network, it 
does not come with any associated cost, and transfers are not a big 
inconvenience for the 3rd party nodes.

The only downside is the fact that the nodes in the network which are not 
actively transfering will occasionally have to donate a part of their 
bandwidth. The users could configure a bandwidth limit amount in such a way as 
to not negatively impact their internet usage. Other data, that is not big, but 
should be mentioned, is the frequent background traffic required for node 
discovery messages and forwarding search requests and replies. A future version 
of this paper will be released, in which this background traffic and passive 
transfer assistance are measured in a simulation.

Another downside is the fact that a file cannot be downloaded from multiple 
sources at a time, similar to BitTorrent, in order to reduce traffic for 
individual file distributors and increase transfer speeds. A future version of 
the protocol might include this functionality.

SAND could work really well is low to mid income countries where piracy and 
cheap high-speed internet broadband are common, like in Romania for example. As 
a side note, in Romania, these circumstances allowed for many citizens to learn 
technical skills and be lifted out of poverty, while the country is now one of 
the leading in technology workers per capita \cite{romaniasoftwaredev}.

Clandestine software and speciality literature usage without legal consequences 
may very well help people in other underdeveloped countries, like those in 
Africa. We could consider this as a means of the western world to pay back a 
bit for the continuing neocolonial practices.
