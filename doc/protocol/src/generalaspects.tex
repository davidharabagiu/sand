SAND is an application layer network protocol on top of the TCP/IP stack. It 
makes use of a worldwide P2P network to allow anonymously distributing files on 
the internet.

The participants in the network are:
\begin{itemize}
    \item Peers: Normal nodes which transfer files between them
    \item DNL Nodes: Nodes which serve the role of providing an initial set of 
neighbors for new peers. The addresses of these nodes should be publicly known
\end{itemize}

It can additionally use UPnP to programmatically add a port mapping to the 
Internet Gateway Device. This will allow incoming messages on the port on which 
the client application is listening if the device is sitting behind a NAT 
capable gateway. Other mechanisms for NAT traversal can be used as well, 
depending on the protocol implementation.

The protocol consists of a number of messages (also called requests), which may 
or may not receive a reply, depending on the type of the message. The request 
contains a 1-byte code which identifies the request type. The rest of the 
message is occupied by auxiliary data which are specific to each request type. 
The reply format is similar, but instead of a request code, it contains a 
1-byte status code. The status code can be interpreted in various ways, 
depending on the request type.

\begin{figure}[H]
    \centering
    \scalebox{.33}{\includegraphics{figures/fig1}}
\end{figure}

Request codes are divided in 8 categories, each having 32 values:
\begin{itemize}
    \item Codes 0 - 31: Reserved / experimental
    \item Codes 32 - 63: Peer discovery
    \item Codes 64 - 95: File searching
    \item Codes 96 - 127: File transfer
    \item Codes 128 - 255 (4 categories): Unused
\end{itemize}
